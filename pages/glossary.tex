% % refer to https://en.wikibooks.org/wiki/LaTeX/Glossary for acronyms and glossary entries

% \newacronym[shortplural={D$_{\text{dye}}$}, longplural={donor dye, ex. Alexa 488}]{ddye}{D$_{\text{dye}}$}{donor dye, ex. Alexa 488}

% \newacronym[description={\glslink{r0}{F\"{o}rster distance}}]{R0}{$R_{0}$}{F\"{o}rster distance}

% \newglossaryentry{r0}
% {
%   name=\glslink{R0}{\ensuremath{R_{0}}},
%   text=F\"{o}rster distance,
%   description={F\"{o}rster distance, where 50\% ...}, 
%   sort=R
% }

% \newglossaryentry{kdeac}
% {
%   name=\glslink{R0}{\ensuremath{k_{DEAC}}},
%   text=$k_{DEAC}$, 
%   description={is the rate of deactivation from ... and emission)}, 
%   sort=k
% }

% \newacronym[shortplural={TUM}, longplural={Technical University of Munich}]{tuma}{TUM}
% {Technical University of Munich}

% \newglossaryentry{tum}
% {
%   name = TUM,
%   description = {A university in Germany, Bavaria, in the city of Munich},
%   plural = TUM
% }

% \newglossaryentry{computer}
% {
%   name=computer,
%   description={is a programmable machine that receives input,
%               stores and manipulates data, and provides
%               output in a useful format}
% }

% \newacronym{uria}{URI}{Uniform Resource Identifier}
% \newglossaryentry{uri}{
%     name = URI,
%     description = {A unique sequence of characters that identifies a logical or physical resource used by web technologies},
%     plural = URIs
% }