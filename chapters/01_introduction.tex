% !TeX root = ../main.tex
% Add the above to each chapter to make compiling the PDF easier in some editors.

\chapter{Introduction}
\label{chapter:introduction}

\section{Motivation}
For the interactive theorem prover \emph{Isabelle}~\parencite{lpi}, the \emph{jEdit IDE}~\parencite{jedit} has been used as the basis to build the \emph{Isabelle/jEdit prover IDE}~\parencite{isabelle_jedit}. To provide the same functionality in a more modern and popular code editor, the Isabelle extension for \emph{VSCode}~\parencite{vscode} (short for Visual Studio Code) was introduced. Although a good portion of the features supported in jEdit are now available for VSCode, the extension has not seen much adoption from users. In its current state, \emph{Isabelle/VSCode}~\parencite{pide} is no match for its jEdit counterpart. This is due to many issues which have cropped up with the extension. The issues that mainly stop it from being a viable alternative are:
\begin{itemize}
    \item Performance issues related to the rendering of mathematical symbols.
    \item Ergonomic issues while working with mathematical symbols.
    \item Missing input options for mathematical symbols, which are present in jEdit.
    \item Missing syntax highlighting for different semantic editor perspectives.
\end{itemize}

The main goal of this project is to alleviate these issues so that Isabelle/VSCode becomes an attractive option to users. In order to achieve this, the focus has been set on the following objectives:
\begin{itemize}
    \item Add first-class support for Unicode characters. This should remove the performance and ergonomic issues with the mathematical symbols.
    \item Provide abbreviation and auto-completion support for the extension.
    \item Add the missing syntax highlighting for the semantic editor perspectives.
\end{itemize}

Of course, some features, which are available in jEdit, would still be missing. Nonetheless, this project should be a major stride towards wider user adoption for Isabelle/VSCode.

\section{Outline}
\Cref{chapter:background} starts the thesis with a short rundown of preliminary information about Isabelle, VSCode, and the Language Server Protocol.

In \Cref{chapter:implementation}, we describe the implementation details of the project. We start with adding the syntax highlighting in \Cref{section:state-panel}. Then, in \Cref{section:symbols}, we deal with the performance issues related to the mathematical symbols. In the last section of the chapter, we add new ways for a user to input mathematical symbols.

In \Cref{chapter:evaluation}, we evaluate the effect of our changes on the performance of the extension. We also compare the performance of the existing solution and the new solution.


\Cref{chapter:conclusion} concludes the thesis with a summary of the goals reached during the course of the thesis and some closing remarks on the project.

\Cref{chapter:future_work} describes new features which can be added to Isabelle/VSCode by future projects. Moreover, it lists other code editors for which the work done here may become relevant.